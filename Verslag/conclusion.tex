\section{conclusion}
The primary goal of this report was to find a relation between the trap stiffness of an optical trap and the output power of the laser used in that optical trap. The predicted linear relation was measured and found the results are shown in the table below.\\
Rewriting the MATLAB script using Python was a partial success, the symmetry centre finding function and the subpixel interpolation function were successfully implemented in Python but the tracking function currently uses a slow and not very useful interpolation method which results in an undesirable offset and delay in the predicted symmetry centre. With a properly functioning interpolation function the python script will without a doubt yield the right results.\\
A second method for deriving the trap stiffness was also tried to be implemented. Using the spread of the symmetry centres of the datasets to try and create an covariance matrix. We were unable to finish this method but were able to show that there is indeed a relation between the ellipse size and the laser power output. This relation was found to be an exponential.\\


\vspace{-0.5cm}
\begin{table}[h!]
    \centering
    \begin{tabular}{|l|l|l|l|}
        \hline
        coefficient for & $a_x$ {[}$pN/(nm\cdot mW)${]} & $a_y$ {[}$pN/(nm \cdot mW)${]} & $a_{tot}$ {[}$pN/(nm \cdot mW)${]} \\ \hline
        dataset 1       & $1.382\cdot 10^{-6}$          & $1.280 \cdot 10^{-6}$          & $1.740 \cdot 10^{-4}$              \\ \hline
        dataset 2       & $1.478 \cdot 10^{-5}$         & $3.974 \cdot 10^{-6}$          & $1.551 \cdot 10^{-5}$              \\ \hline
    \end{tabular}
    \caption{The results of the linear fits for the trap stiffness as a function of laser output power. The shown $a$-factors satisfy the least squares fit of the function: $k_i = a_i \cdot P$ }
\end{table}