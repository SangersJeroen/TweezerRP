\section{Introduction}

An optical trap, or optical tweezers, is a technique that is frequently used in molecular biology to study particles at the micro- and nanometre scale. By trapping a particle in a focussed laser beam, the particle is limited in movement. This allows the user to study microscopic manipulations and measurements on microscopic particles and therefore proves to be very useful in the field of biophysics. Examples of which are sorting of cells, unzipping of DNA and enzyme interactions \cite{MIT}\cite{Velthuis}.
In order to perform quantitative measurements using optical tweezers, it is vital to know what force it exerts on the particle. This force is defined by the trap constant and is dependent on various parameters such as the particle size and laser power. 
The aim of this report is to find the relation between the laser power and the trap constant. Secondly this report serves to get familiarised with the optical tweezers technique and to investigate its limitations and possibilities. 
Due to the COVID-19 virus no experiments were carried out for this report and data of previous experiments by other students is used. This data contains images of a trapped bead for laser beams with different powers. For each laser intensity, there is a set of images at fixed time intervals such that the movement of the bead can be studied. For this report, the images are processed by a MATLAB algorithm which calculates the trap constant in two perpendicular directions. Using these two values, the overall trap constant can also be determined. The relation between each trap constant and the laser power is found using a distance regression method and the theoretical linear dependence.
In section 2 the theory regarding the report will be described followed by the experimental method in section 3. The results and discussion can be found in section 4. Lastly the conclusions in section 5.








