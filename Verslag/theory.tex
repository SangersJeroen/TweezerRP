\Section{Theory}

 the optical tweezers technique was used in the experiment that corresponds with the data that is used in this report. The basic principles of optical tweezers and the theory for further calculations is described in this section.
 
 
\subsection*{Optical trap}

To understand the working principle of optical tweezers we consider a spherical dielectric particle, a bead, in a coherent light beam with a symmetrical intensity gradient such as in figure \ref{fig_optic_trap_1}. The light beam will exert a force on the bead due to radiation pressure \cite{ashkin}. This force is a result of the momentum that photons carry and will be directly proportional to the light intensity. \
We now consider two rays of light that reach the bead symmetrical with respect to its centre. Due to the spherical symmetric shape, the light rays will be refracted by the dielectric particle at the same rate, but in opposite directions (see figure \ref{fig_optic_trap_1}). Both light rays will, given the change in direction of the light and the third law of Newton, exert a force on the bead. The light ray with the higher intensity will, however, exert a greater force. If the intensity gradient is greater in the centre of the light beam, such as in figure \ref{fig_optic_trap_1}, this would lead to a net force pointing in the direction of the symmetry axis of the light beam. This force would trap the bead to the centre on the beam. \
In the case of a the beam of light being focussed such as in figure \ref{fig_optic_trap_2}, the bead would not only be trapped in the direction perpendicular to the beam axis, but also in the direction of the axis. This is also a result of the change of the refraction of light exerting a force on the bead (see figure \ref{fig_optic_trap_2}). However, to light scattering, the bead is in the axial direction trapped slight behind the waist of the light beam.\cite{Sheavitz} \
Note that this explanation of optical tweezers is only valid if the dimensions of the bead are much greater than the wavelength of the light. For only in this situation we can use ray optics treatment. In the case of a relatively small particle, the particle is also trapped in the same manner.  \
According to Shaevitz (2006), 'for small motions of a bead near the center of an optical trap, the forces acting on the bead approximate a zero rest–length, linear spring at the trapping center.' Therefore, for small motions of the bead, the stored energy in the optical 'spring' is $1/2 k_{trap} <x^2>$ with $k_{trap}$ a constant defining the strength of the optical trap and $<x^2>$ the variance in the motion. According to the equipartition theory, the energy of the Brownian motion of a particle is given by $1/2 k_b T$ with $k_b$ the Boltzmann constant and $T$ the temperature.\cite{Shaevitz} Equating the two energies yields:

\begin{equation}
	\label{eq_k_trap}
	k_{trap} = \frac{k_B T}{<x^2>}
\end{equation}

From this we can conclude that by following the position of the bead over time, it is possible to find the the value of $k_{trap}$. \
In real life, the three-dimensional case, it is possible to find the $k_{trap}$ for a single dimension by finding the variance along that dimension. However, if one would like to describe the whole system, it is necessary to find the covariance matrix of the motion. This way, it is possible to find the covariance 


